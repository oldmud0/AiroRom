% Copyright (c) 2015 Benito Palacios Sánchez - All Rights Reserved.
% Esta obra está licenciada bajo la Licencia Creative Commons Atribución 4.0
% Internacional. Para ver una copia de esta licencia, visita
% http://creativecommons.org/licenses/by/4.0/.

\section{Bibliografía}
\begin{frame}{Principales references bibliográficas}
\begin{itemize}
    \item<1-> Martin Korth. \textit{GBA/NDS Technical Info}. {\footnotesize\url{http://problemkaputt.de/gbatek.htm}}.

    \item<2-> GBATemp. \textit{NDS - ROM Hacking and Translations}. {\footnotesize\url{http://gbatemp.net/forums/nds-rom-hacking-and-translations.41}}.

    \item<3-> ROMHacking.net. {\footnotesize\url{http://www.romhacking.net}}.

    \item<4-> The REDO compendium. \textit{Reverse Engineering for Software Maintenance}. 1993.

    \item<5-> Eldad Eilam. \textit{Reversing. Secrets of Reverse Engineering}. 2005.

    \item<6-> Andrew Huang. \textit{Hacking the Xbox. An Introduction to Reverse Engineering}. 2003.
\end{itemize}
\end{frame}

\begin{frame}
    \begin{alertblock}{Andew Huang - Hacking the Xbox. An introduction to Reverse Engineering}
        In general, I hack because it is quite satisfying to know that somebody's life was made better by something I built. I feel it is my obligation to apply my talents and return to society what it has given me. I also enjoy the challenge of exploration. I want to understand electronics as deeply as I can. Black boxes frustate me; nothing gets my curiosity going more than a box that I'm not allowed to open or understand. As a result, I have a fiduciary interest in cryptography and security methods.
    \end{alertblock}
\end{frame}
