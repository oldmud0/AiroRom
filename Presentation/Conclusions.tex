% Copyright (c) 2015 Benito Palacios Sánchez - All Rights Reserved.
% Esta obra está licenciada bajo la Licencia Creative Commons Atribución 4.0
% Internacional. Para ver una copia de esta licencia, visita
% http://creativecommons.org/licenses/by/4.0/.

\section{Conclusiones}
\subsection{}

\begin{frame}{Conclusiones}
\begin{itemize}
    \item<+-> Identificar problemas no tratados en la literatura.

    \item<+-> Organizar trabajo en tareas e investigarlas.

    \item<+-> Desarrollar software en C\# y python para ayudar la investigación.

    \item<+-> Aprender conceptos de bajo nivel en software y hardware incluyendo el lenguaje ensamblador \texttt{ARM}.

    \item<+-> Diseñar metodologías de ingeniería inversa.

    \item<+-> Diseñar estrategias para capturar paquetes de datos de videojuegos.

    \item<+-> Analizar 21 juegos.

    \item<+-> Aprender \LaTeX.
\end{itemize}
\end{frame}

\begin{frame}{Trabajo futuro}
\begin{itemize}
    \item<+-> Estudiar seguridad en videoconsolas y sus \textit{exploits}.

    \item<+-> Estudiar algoritmos de integridad en partidas de guardado.

    \item<+-> Estudiar mecanismos anti-copia implementados digital y físicamente en videojuegos.

    \item<+-> Estudiar protocolo de micropagos en videojuegos.

    \item<+-> Analizar aplicaciones de ordenador (\textit{Steam}).

    \item<+-> Implementar mecanismos estudiados.

    \item<+-> Desarrollar un explorador de juevos avanzado, con reconocimiento de formatos y algoritmos.

    \item<+-> Desarrollar un depurador de código remoto.
\end{itemize}
\end{frame}
