% Copyright (c) 2015 Benito Palacios Sánchez - All Rights Reserved.
% Esta obra está licenciada bajo la Licencia Creative Commons Atribución 4.0
% Internacional. Para ver una copia de esta licencia, visita
% http://creativecommons.org/licenses/by/4.0/.

\section{Introducción}
\subsection{}

\begin{frame}{Motivación}
\begin{columns}
    \begin{column}{0.5\textwidth}
    \begin{wideitemize}
        \item<1-> Los videjuegos son una clave de nuestra cultura actual.

        \item<2-> Su industria es la segunda con más ganancias.

        \item<3-> Preocupación por protección anti-copias, derechos de autor, trampas.
    \end{wideitemize}
    \end{column}

    \begin{column}{0.5\textwidth}
        \only<1>{
            \includefigure{Estadísticas sobre jugadores en EE. UU. \scriptsize{}Fuente: \url{http://www.esrb.org}}{imgs/gamer_stats.png}
        }

        \only<2->{
            \includefigure{Estadísticas sobre la industria de videojuegos en EE. UU. \scriptsize{}Fuente: \url{http://www.esrb.org}}{imgs/game_ind_stats.png}
        }
    \end{column}
\end{columns}
\end{frame}

\begin{frame}{\textit{ROM Hacking}}
    \begin{block}{Ingeniería inversa}
        La ingeniería inversa es el proceso de analizar un sistema para identificar sus componentes y relaciones y, crear una representación del sistema en otro formato o a un nivel más alto de abstracción.
    \end{block}

    \begin{block}{\textit{ROM Hacking}}<2->
        Ingeniería inversa sobre videojuegos. El nombre viene realizar modificaciones (\textit{hacks}) sobre juegos que suelen distribuirse en memorias de solo lectura (\textit{Read Only Memory}).
    \end{block}
\end{frame}
