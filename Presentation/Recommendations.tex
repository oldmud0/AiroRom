% Copyright (c) 2015 Benito Palacios Sánchez - All Rights Reserved.
% Esta obra está licenciada bajo la Licencia Creative Commons Atribución 4.0
% Internacional. Para ver una copia de esta licencia, visita
% http://creativecommons.org/licenses/by/4.0/.

\section{Recomendaciones}
\subsection{}

\begin{frame}{Seguridad en ficheros}
\note<1>[item]{Es de señalar la revelancia de estas recomendaciones, pues hasta la fecha no existe reportado nada similar en la literatura.}

\uncover<1->{Acceso a ficheros:}
\begin{itemize}
    \item<2-> Empaquetar ficheros.
    \note<1>[item]{Guitar Hero: On Tours y Profesor Layton y la Llamada del Espectro.}
    \begin{itemize}
        \item<3-> Implementaciones largas (1.900 líneas).
        \item<4-> Compresión diferente en cada cabecera.
    \end{itemize}

    \item<5-> Ofuscar nombre de ficheros y directorios.
    \note<1>[item]{Pokémon Blanco y Negro}
\end{itemize}

\uncover<6->{Cifrado en textos e imágenes:}
\begin{itemize}
    \item<7-> Cifrado \texttt{XOR}.
    \begin{itemize}
        \item<8-> No usar claves estáticas.
    \end{itemize}

    \item<9-> Codificación de caracteres no estándar.
    \begin{itemize}
        \item<10-> Desordenar caracteres.
        \item<11-> Cifrado del archivo de tipografía.
    \end{itemize}

    \item<12-> Nuevos formatos frente a cifrado.
\end{itemize}

\uncover<+->{Contenido con derechos de autor sin protección.}
\end{frame}

\begin{frame}{Seguridad en comunicaciones}
\begin{wideitemize}
    \item<+-> HTTPS vs HTTP.
    \note<1>[item]{Ataques man-in-the-middle.}
    \begin{itemize}
        \item<+-> Cerrar puertos de servidores.
        \item<+-> Diseño del protocolo.
    \end{itemize}

    \item<+-> Autenticación con contraseña vs reto.
    \begin{itemize}
        \item<+-> Comprobar la contraseña.
    \end{itemize}

    \item<+-> Cifrado y comprobación de integridad en descargas.
    \begin{itemize}
        \item<+-> Proteger activación de contenido descargado.
        \note<1>[item]{Añadir contraseña a la base de datos.}
    \end{itemize}

    \item<+-> Transmisión segura de código entre dispositivos.
\end{wideitemize}
\end{frame}
