% Copyright (c) 2015 Benito Palacios S�nchez - All Rights Reserved.
% Esta obra est� licenciada bajo la Licencia Creative Commons Atribuci�n 4.0
% Internacional. Para ver una copia de esta licencia, visita
% http://creativecommons.org/licenses/by/4.0/.

\chapter{Conclusiones}
% De qu� va el cap�tulo
A lo largo del trabajo se han analizado una serie de juegos, estudiando los mecanismos de protecci�n que implementan y se�alando sus carencias y debilidades.
Todos los objetivos propuestos han sido alcanzos, analizando as� un total de 23 juegos de los 20 previstos.
De estos, se han mencionado en esta memoria un total de 14 y documentando el resto en la wiki del repositorio del trabajo\footnote{\url{https://github.com/pleonex/airorom/wiki/Mecanismos-a-investigar}}.

Se plantearon dos objetivos opcionales que por falta de tiempo no han podido cumplir.
El primero de ellos fue desarrollar un nuevo programa de exploraci�n de juegos.
Tras analizar las necesidades de este trabajo y el tiempo previsto, se decidi� optar por usar \textit{Tinke} que cumpl�a los requirimientos b�sicos (explorar archivos, reconocer formatos y visualizar im�genes en formatos est�ndar).
El segundo fue crear un depurador de c�digo para \acl{NDS} llamado \textit{NitroDebugger}\footnote{\url{https://github.com/pleonex/NitroDebugger}} del cual, se ha terminado su n�cleo y faltar�a crear una interfaz gr�fica y desensamblador.
Este proyecto se present� al \textit{Certamen de Proyectos Libres de la UGR 2014}, a pesar de no estar entre los finalistas debido a estado de completitud, recibi� buenas cr�ticas del jurado\footnote{\url{https://goo.gl/g8xdWZ}}.

En cuanto a objetivos acad�micos, se ha sido capaz de identificar un programa no tratado, realizar una selecci�n de objetivos e investigarlos.
Se ha realizado \textit{software} adicional para complementar la investigaci�n y se ha usado un control de versiones.
Durante el trabajo se han tocado conceptos de bajo nivel de \textit{software} y \textit{hardware} estudiando protocolos y componentes de la \acl{NDS}, as� como su lenguaje ensamblador, \texttt{ARM}.
Adem�s, se han dise�ado estrategias para estudiar las comunicaciones inal�mbricas como ha sido modificar el emulador DeSmuME para guardar los paquetes capturados y realizar un dise�o \textit{man-in-the-middle} para capturar el tr�fico de las aplicaciones m�viles.
Finalmente, se ha aprendido el lenguaje \LaTeX para la redacci�n de esta memoria.

En este cap�tulo se mencionar�n y resumir�n, propiendo en algunos casos m�todos para fortalecerlas.
En la segunda secci�n se mencionan temas en los que este trabajo podr�a profundizar en el futuro.

\section{Resultados y recomendaciones}
% Resumir cada fallo / seguridad y a�adir recomendaci�n

\section{Trabajo futuro}
Los estudios de este trabajo han pretendido resumir los mecanimos m�s frecuentes encontrados en los juegos de \ac{NDS} as� como, se�alas las carencias de las plataformas m�viles.
A continuaci�n se ofrece una lista sobre t�picos en los que los que se podr�a profundizar.
