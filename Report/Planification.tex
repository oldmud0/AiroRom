% Copyright (c) 2015 Benito Palacios S�nchez - All Rights Reserved.
% Esta obra est� licenciada bajo la Licencia Creative Commons Atribuci�n 4.0
% Internacional. Para ver una copia de esta licencia, visita
% http://creativecommons.org/licenses/by/4.0/.

\chapter{Requisitios, planificaci�n y metodolog�a}
\label{sec:requirements}

\section{Metodolog�a}

\subsection{Claves XOR en texto plano}
\label{sec:met-xor-key}

\subsection{B�squeda de algoritmos sobre textos}
\label{sec:met-search-tex}
% TODO:
Tras una b�squeda de las frases de di�logos iniciales sobre todo el archivo binario del juego no se encuentran resultados.
Se prob� en las codificaciones m�s frecuentes de esta consola como son ASCII y UNICODE sin resultado.
Esto indica que bien el archivo que contiene los textos est� comprimido o cifrado.

El siguiente procedimiento fue extraer la memoria RAM del juego justo en el momento de mostrar ese di�logo, pues deber�a estar almacenada esa frase para poder ser mostrada en pantalla.
Esto se puede realizar gracias al emulador DeSmuME mencionado en el cap�tulo~\ref{sec:requirements}.
Una vez extra�do el archivo binario con la memoria RAM, mediante visores hexadecimales se busc� la frase que aparec�a en pantalla usando las codificaciones est�ndar y sin obtener resultado de nuevo.
Dado que la frase ha de estar en la memoria RAM, el problema era por tanto que no estaba usando una codificaci�n est�ndar.
Existen dos procedimientos b�sicos para realizar una b�squeda en estos casos.

El primero consiste en buscar el archivo con la tipograf�a del juego, pues la codificaci�n ser� la misma que el orden con el que aparecen los caracteres en ella.

El segundo procedimiento posible es el de desarrollar un programa de b�squeda diferencial. Con motivo de este trabajo en el cap�tulo~\ref{sec:requirements} se explica el programa \textit{RelativeSearch} que se llev� a cabo. Este tipo de \textit{software} se puede usar siempre y cuando el orden de los caracteres de un mismo grupo (por ejemplo letras min�sculas) corresponda a est�ndares como ASCII. Si se hubiese seguido un orden aleatorio en la codificaci�n propietaria este no servir�a quedando la primera opci�n como �nica alternativa.

Tras usar cualquiera de los dos procedimientos anteriormente descritos, se puedo encontrar una coincidencia en la memoria RAM del di�logo. El siguiente paso consisti� en depurar el juego para hallar el algoritmo de descifrado o descompresi�n de textos. Para ello usando los programas descritos en el cap�tulo~\ref{sec:requirements} de depuraci�n, se puso un punto de interrupci�n en dicha posici�n.

Cabe destacar que esta posici�n es din�mica por lo que cambia cada vez que se inicia el juego pues depende de muchas variables. Para sortear este problema se hizo uso de los \textit{savestates} del emulador, es decir guardados de memoria del juego que permiten ir a cualquier momento de una ejecuci�n. Gracias a esta caracter�stica se puede realizar un guardado justo antes de que el juego guarde un juego de forma que siempre que se use ese \textit{savestate} encontraremos el texto en la misma posici�n. El punto de interrupci�n hizo parar el emulador justo en las instrucciones m�quinas que estaban realizando, en este caso, el descifrado.
